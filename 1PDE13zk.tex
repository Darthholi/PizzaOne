%% LyX 2.0.5.1 created this file.  For more info, see http://www.lyx.org/.
%% Do not edit unless you really know what you are doing.
\documentclass[english,a4paper,english,czech,12pt]{article}
\usepackage[utf8]{inputenc}
\usepackage{babel}
\usepackage{amsmath}
\usepackage{amssymb}
\usepackage{esint}
\usepackage[unicode=true,
 bookmarks=true,bookmarksnumbered=false,bookmarksopen=false,
 breaklinks=false,pdfborder={0 0 1},backref=false,colorlinks=false]
 {hyperref}
\hypersetup{pdftitle={1PDE13ZS},
 pdfauthor={Martin Holi Holecek}}

\makeatletter
%%%%%%%%%%%%%%%%%%%%%%%%%%%%%% Textclass specific LaTeX commands.
\newenvironment{lyxcode}
{\par\begin{list}{}{
\setlength{\rightmargin}{\leftmargin}
\setlength{\listparindent}{0pt}% needed for AMS classes
\raggedright
\setlength{\itemsep}{0pt}
\setlength{\parsep}{0pt}
\normalfont\ttfamily}%
 \item[]}
{\end{list}}

%%%%%%%%%%%%%%%%%%%%%%%%%%%%%% User specified LaTeX commands.
\usepackage{amsthm,amsmath,amssymb,amscd,mathrsfs,url}
%\usepackage[czech]{babel}
\usepackage{verbatim}
\usepackage{showkeys}
\usepackage{xcolor}
\usepackage{stackengine}
\usepackage{lipsum}
\usepackage{comment}
%%\usepackage{MNsymbol} NIKDY!
\usepackage{graphicx}
\usepackage{tikz}
\usepackage{marginnote}
\usepackage{esint}
\setlength\textwidth{145mm}
\setlength\textheight{247mm}
\newcommand{\opdiv}{\mathop{\mathrm{div}}}
\newcommand{\oprot}{\mathop{\mathrm{rot}}}

\newcommand*\mnote[3][0pt]{%
  \if l#2\reversemarginpar\def\pointer{\filledmedtriangleright}%
    \def\stackalignment{r}\fi%
  \if r#2\normalmarginpar\def\pointer{\filledmedtriangleleft}%
    \def\stackalignment{l}\fi%
  \marginpar{%
    \topinset{%
      \scalebox{1.5}{\textcolor{blue}{$\pointer$}}}{%
      \belowbaseline[-1.5\baselineskip-#1]{%
        \stackengine%
          {-5pt}%
          {\fcolorbox{blue}{white}{\parbox{1.8cm}%
            {\vspace{3pt}\raggedright#3}}}%
          {~\colorbox{white}{\sffamily Pozn.}}%
          {O}%
          {l}%
          {F}%
          {F}%
          {S}%
        }%
      }{%
      3ex+#1}{%
      -2ex}%
  }%
}

\makeatother

\begin{document}

\title{P{\footnotesize{izza}} D{\footnotesize{omů}} R{\footnotesize{ychle}}
I ZS}

\maketitle

\subsection*{Sobolevův prostor $W^{k,p}(\Omega)$}


\paragraph*{Multiindex}

$\alpha\in\mathbb{N}_{0}^{d}$, tedy vektor přirozených čísel s nulou.
Výška multiindexu je součet jeho složek.


\paragraph*{Množina}

$\Omega$ je otevřená (kvůli definici slabé derivace).


\paragraph*{Derivační operátor $D^{\alpha}$:}

funkce $u\in C^{k}(\Omega),\forall\alpha,|\alpha|\leq k$ : 

\[
D^{\alpha}u=\frac{\partial^{|\alpha|}u}{\partial x^{\alpha}}=(\frac{\partial^{\alpha_{1}}}{\partial x_{1}^{\alpha_{1}}}...\frac{\partial^{\alpha_{d}}}{\partial x_{n}^{\alpha_{d}}})u
\]



\paragraph*{Slabá derivace}

Nechť $\Omega\subset\mathbb{R}^{d}$ otevřená, $u\in L_{loc}^{1}(\Omega)$,
pak $v\in L_{loc}^{1}(\Omega)$ je $\alpha$-slabá derivace $u$ právě
tehdy když

\[
\forall\varphi\in\mathcal{D}(\Omega):\,\int_{\Omega}v\varphi\, dx=(-1)^{|\alpha|}\int_{\Omega}uD^{\alpha}\varphi\, dx
\]


$\mathcal{D}(\Omega)$ - hladké funkce s kompaktním nosičem na $\Omega$.

Pokud $u\in C^{k},|\alpha|\leq k\Rightarrow v=D^{\alpha}u$ je $\alpha$-slabá
derivace. Nazvěme $D^{\alpha}u$ slabou derivací vždy.


\paragraph*{Sobolevův prostor}

Nechť $\Omega\subset\mathbb{R}^{d}$ otevřená, $k\in\mathbb{N},p\in[1,\infty]$,
definujeme $W^{k,p}(\Omega)$:

\[
W^{k,p}(\Omega):=\{u\in L^{p}(\Omega),\forall|\alpha|\leq k:\, D^{\alpha}u\in L^{p}(\Omega)\}
\]


S normou pro $p<\infty$:

\[
||u||_{k,p}=(\sum_{|\alpha|\leq k}||D^{\alpha}u||_{p}^{p})^{1/p}
\]


a pro $p=\infty$:

\[
||u||_{k,\infty}=\sum_{|\alpha|\leq k}||D^{\alpha}u||_{\infty}
\]

\begin{itemize}
\item $W^{k,p}(\Omega)$ je normovaný lineární prostor.
\item $W^{k,p}(\Omega)$ je Banachův prostor
\item $W^{k,p}(\Omega)$ je Separabilní $\Leftrightarrow p\in[1,\infty)$
\item $W^{k,p}(\Omega)$ je Reflexivní $\Leftrightarrow p\in(1,\infty)$
\item $W^{k,2}(\Omega)$ je Hilbertův $\Leftrightarrow p=2$, $(u,v)_{k,2}=\sum_{|\alpha|\leq k}\int_{\Omega}D^{\alpha}u\, D^{\alpha}v\, dx$
\end{itemize}

\subsection*{Holderovsky spojité funkce}

Funkce je $\lambda$-holderovsky spojitá, na uzávěru otevřené $\Omega$: 

\[
C^{0,\lambda}(\bar{\Omega})\,:=\{u\in C(\bar{\Omega}),\exists c>0\,:\,\forall x,y\in\Omega,\, x\neq y:\,\frac{|u(x)-u(y)|}{|x-y|^{\lambda}}\leq c\}
\]


S normou:

\[
||u||_{C^{0,\lambda}}=||u||_{C}+\sup_{x\neq y}\frac{|u(x)-u(y)|}{|x-y|^{\lambda}}
\]


Arzela-Ascoli: $C^{0,\alpha}\hookrightarrow\hookrightarrow C^{0,\beta}$
pro $\alpha>\beta$ (``větší do menšího'').

{\footnotesize{Pozor na to, že $x\neq y$, $\bar{\Omega}$ uzávěr!}}{\footnotesize \par}


\subsection*{Funkce třídy $C^{k,\lambda}$}

$a\in C^{k,\lambda}(\bar{\Omega})\iff a\in C^{k}(\bar{\Omega})\,\&\,\forall|\beta|\leq k\, D^{\beta}u\in C^{0,\lambda}(\bar{\Omega})$
\begin{itemize}
\item {\footnotesize{Spojitá ... $C^{0,0}$}}{\footnotesize \par}
\item {\footnotesize{Lipschitzovská ... $C^{0,1}$}}{\footnotesize \par}
\end{itemize}

\subsection*{Množina třídy $C^{k,\lambda}$}

Otevřená omezená $\Omega$ je třídy $C^{k,\lambda}$, $k\in\mathbb{N}_{0},\lambda\in[0,1]$,
právě tehdy, když existuje $M\in\mathbb{N}$ koordinátových systémů
a čísla $\alpha,\beta\in\mathbb{R}^{+}$, a funkce $a_{n}:\,(-\alpha,\alpha)^{d-1}\rightarrow\mathbb{R},\, a_{n}\in C^{k,\lambda},\, n=1...M$,
že platí při značení $x=(x_{1}...x_{d})=(x',x_{d})$ následující:

\[
\Omega_{n}^{+}:=\{x,|x'|<\alpha,a_{n}(x')<x_{d}<a_{n}(x')+\beta\}\subset\Omega
\]


\[
\Omega_{n}^{-}:=\{x,|x'|<\alpha,a_{n}(x')-\beta<x_{d}<a_{n}(x')\}\subset(\mathbb{R}^{d}\backslash\bar{\Omega})
\]


\[
\delta_{n}\Omega:=\{x,|x'|<\alpha,a_{n}(x')=x_{d}\}\subset\delta\Omega
\]


\[
\bigcup_{n=1}^{M}\delta_{n}\Omega=\delta\Omega
\]


{\footnotesize{Nezapomenout z jakých prostorů jsou $k,\lambda,\alpha,\beta$
a na def. obor $a_{n}$.}}{\footnotesize \par}


\subsection*{\pagebreak{}}


\subsection*{Lokální aproximace}

Pro $p\in[1,\infty)$, $\Omega\subseteq\mathbb{R}^{d}$ otevřená:
$\forall u\in W^{k,p}(\Omega)\,\exists u^{n}\in C^{\infty}(\Omega)$,
že pro každou $\Omega'\subset\subset\Omega$ posloupnost $u^{n}\longrightarrow u$
konverguje ve $W^{k,p}(\Omega')$.

Dokonce bude platit $u^{n}\in C^{\infty}(\mathbb{R}^{d})$.

{\footnotesize{Zapisuje se také jako ``$u^{n}\longrightarrow u$
konverguje ve $W_{loc}^{k,p}(\Omega)$''.}}{\footnotesize \par}


\subsubsection*{Vlastnosti konvolučního zhlazení}

Fakt - konvolučně zhlazená funkce - pro $u\in L^{p}(\mathbb{R}^{d})$
máme

$u_{\epsilon}:=\rho_{\epsilon}*u=\int_{\mathbb{R}^{d}}\rho_{\epsilon}(x-y)u(y)\, dy$
a ta jde k původní funkci - $||u_{\epsilon}-u||_{L^{p}(\mathbb{R}^{d})}\rightarrow0$.

Konvoluční zhlazení navíc zvětší o $\epsilon$ nosič, tedy je potřeba
volit dostatečně malé, abychom v důkazech nevybočili z množin.

Často při konvolučním zhlazení potřebujeme funkci definovat mimo množinu,
kde je definvoaná původně, dodefinováváme tedy nulou. Zhlazení pak
zhladí funkci, která je nespojitá se skokem na hranici na $C^{\infty}$
funkci.


\subsubsection*{Lemma o záměně zhlazení a $D^{\alpha}$}

$\Omega\subset\mathbb{R}^{d}$ otevřená, pak pro každé $u\in W_{(loc)}^{k,1}(\Omega)$
platí skoro všude v $\Omega_{\epsilon}:=\{x\in\Omega,dist(x,\delta\Omega)>\epsilon\}$
rovnost $D^{\alpha}(u{}_{\epsilon})=(D^{\alpha}u)_{\epsilon}$, tu
čteme (dokážeme) tak, že slabá derivace $u{}_{\epsilon}$ na levé
straně je pravá strana.


\paragraph*{Důkaz }

- přehodíme pomocí perpartes a fubiniho derivaci z $\varphi$ přez
zhlazovací jádro na $u$.

\[
\int_{\Omega}u_{\epsilon}(x)D^{\alpha}\varphi(x)\, dx=\int_{\Omega}\int_{\Omega}\rho_{\epsilon}(x-y)u(y)\, dy\, D^{\alpha}\varphi(x)\, dx=
\]


...a fubi:

\[
=\int_{\Omega}\int_{\Omega}\rho_{\epsilon}(x-y)D^{\alpha}\varphi(x)\, dx\, u(y)\, dy=
\]


...perpartes pro dvě hladké funkce s kompaktním nosičem (proto žádné
okrajové členy):

\[
=(-1)^{|\alpha|}\int_{\Omega}\int_{\Omega}D_{x}^{\alpha}\rho_{\epsilon}(x-y)\varphi(x)\, dx\, u(y)\, dy=
\]


...fubi:

\[
=(-1)^{|\alpha|}\int_{\Omega}\int_{\Omega}D_{x}^{\alpha}\rho_{\epsilon}(x-y)u(y)\, dy\,\varphi(x)\, dx=
\]
...protože je $\rho_{\epsilon}$ symetrická, je to jako derivovat
dle y:

\[
=(-1)^{2|\alpha|}\int_{\Omega}\int_{\Omega}D_{y}^{\alpha}\rho_{\epsilon}(x-y)u(y)\, dy\,\varphi(x)\, dx=
\]


... vyřešíme sudý exponent a pak použijeme znovu perparets, čímž se
přenese derivace k $u$:

\[
=\int_{\Omega}\int_{\Omega}D_{y}^{\alpha}\rho_{\epsilon}(x-y)u(y)\, dy\,\varphi(x)\, dx=(-1)^{|\alpha|}\int_{\Omega}\int_{\Omega}\rho_{\epsilon}(x-y)D_{y}^{\alpha}u(y)\, dy\,\varphi(x)\, dx
\]


\[
=(-1)^{|\alpha|}\int_{\Omega}(D^{\alpha}u)_{\epsilon}(x)\varphi(x)\, dx
\]


Tedy lze zaměnit derivace a zhlazení.


\paragraph*{Jak to použijeme v důkazu?}

Máme zadáno $u\in W^{k,p}(\Omega)$, položíme $u(x)=0$ mimo $\Omega$.
Posloupnost zkonstruujeme jako $u_{\epsilon}=\rho_{\epsilon}*u\,\in C^{\infty}(\mathbb{R}^{d})$
(pro nějaké $\epsilon=\epsilon_{n}\rightarrow0_{+}$).

Platí skutečně požadovaná konvergence? Zafixujeme $\Omega'\subset\subset\Omega$
a najdeme $\epsilon_{0}>0$ takové, aby $\forall x'\in\Omega':\, B_{\epsilon_{0}}(x')\subseteq\Omega$
(tzn zhlazené budou od tohoto $\epsilon_{0}$ kompaktně supportem
v $\Omega$, to je předpoklad lemmatu, proto to děláme). Volbou $\epsilon_{0}$
odstraníme maximálně konečně členů posloupnosti, což neovlivní konvergenci.

Pak v posloupnosti dokážu pro všechny členy $\epsilon<\epsilon_{0}$,
že $||u_{\epsilon}-u||_{k,p(\Omega')}\overset{\epsilon\rightarrow0_{+}}{\rightarrow}0$
- to je definice konvergence ve $W^{k,p}(\Omega')$. 

Nu a to bude splněno, pokud normy všech derivací půjdou k nule:

\[
\Leftrightarrow\forall|\alpha|\leq k:\,||D^{\alpha}(u_{\epsilon})-D^{\alpha}u||_{L^{p}(\Omega')}\rightarrow0
\]


...použijeme lemma a tím přetvoříme ekvivalenci:

\[
\Leftrightarrow\forall|\alpha|\leq k:\,||(D^{\alpha}u)_{\epsilon}-D^{\alpha}u||_{L^{p}(\Omega')}\rightarrow0
\]


A dostali jsme známý fakt pro zhlazené funkce $D^{\alpha}u$.


\subsection*{\pagebreak{}}


\subsection*{Globální aproximace s $C^{\infty}(\Omega)$}

Pro $p\in[1,\infty),\,\Omega\subseteq\mathbb{R}^{d}$ otevřená:

$\forall u\in W^{k,p}(\Omega)\,\exists u^{n}\in C^{\infty}(\Omega)\cap W^{k,p}(\Omega)$,
že posloupnost $u^{n}\longrightarrow u$ konverguje ve $W^{k,p}(\Omega)$.


\paragraph*{Důkaz}

Celou dobu budeme mít na zřeteli cíl, tedy najít posloupnost $u^{\delta}\in C^{\infty}(\Omega)$,
aby platilo $||u^{\delta}-u||_{k,p(V)}\leq\delta$ pro libovolné $V\subset\subset\Omega$.
Toto je ekvivalentní s tvrzením věty, protože norma lze rozepsat na
supréma přez $V\subset\subset\Omega$ a to díky spojité závislosti
na oboru integrace, tedy $||u^{\delta}-u||_{k,p(\Omega)}=\sup_{V\subset\subset\Omega}||u^{\delta}-u||_{k,p(V)}$.
Mějme tedy zadané $\delta$ libovolné, ale pevné.

Nejprve zkonstruujeme otevřené pokrytí množiny takto:

\[
i\in\mathbb{N}:\,\Omega_{i}=\{x\in\Omega,dist(x,\delta\Omega)>1/i\}
\]


$U_{i}=\Omega_{i+5}-\bar{\Omega}_{i}$ budou pásy zhušťující se k
hranici.

Dodefinujeme $U_{0}$ otevřenou, aby platilo $\bigcup_{i=0}^{M}\Omega_{i}=\Omega$.

Použijeme rozklad jedničky $\xi_{j}$ vzhledem k $U_{j}$, můžeme,
protože $U_{i}$ mají konečné počty průniků. Dostaneme také $\forall x\in\Omega\,\exists j,\,\xi_{j}(x)=1$.

Díky úvahám výše máme nástroje na nalezení posloupnosti $u^{\delta}$
od $u$. Nejprve rozložíme zadané $u$ rozkladem jednotky:

$u_{i}:=u\xi_{i}$. Nutno říct, že platí $u(x):=\sum u_{i}(x)$, $u_{i}$
jsou teď funkce, jejichž nosič se přibližuje hranici omegy.

Teď každou $u_{i}$ zhladíme:

$u_{i}^{\epsilon}:=\rho_{\epsilon}*u_{i}$, jsme dost blízko hranice
a proto volíme $\epsilon\leq1/(i+5)$. Protože máme $u_{i}^{\epsilon}\overset{\epsilon\rightarrow0_{+}}{\rightarrow}u$,
nalezneme $\epsilon(\delta,i)$, aby $||u_{i}^{\epsilon(\delta,i)}-u_{i}||_{k,p(U_{i})}\leq\delta/2^{i}$.

Přesně položíme $u^{\delta}:=\sum_{i=0}^{\infty}u_{i}^{\epsilon(\delta,i)}\left(=\sum_{i=0}^{\infty}\rho_{\epsilon(\delta,i)}u_{i}=\sum_{i=0}^{\infty}\rho_{\epsilon(\delta,i)}(u\xi_{i})\right)$,
no a konstruovali jsme to tak, aby $u^{\delta}\in C^{\infty}(\Omega)$
(mimochodem ale neplatí $u^{\delta}\in C^{\infty}(\bar{\Omega})$).

Ověříme, že posloupnost splňuje požadavek $||u^{\delta}-u||_{k,p(V)}\leq\delta$
(tzn že konverguje k zadané $u$):

\[
||u^{\delta}-u||_{k,p(V)}\leq||\sum_{i=0}^{\infty}u_{i}^{\epsilon(\delta,i)}-\sum_{i=0}^{\infty}u_{i}||_{k,p(V)}=
\]
 ...Důvod proč jsme dělali rozklad jednotky, je, že teď na $V$ bude
nenulový jen konečný počet $u_{i}$ (a tedy i $u_{i}^{\epsilon(\delta,i)}$),
na konečný počet funguje trojůhelníková nerovnost:

\[
=||\sum_{i=0}^{M}u_{i}^{\epsilon(\delta,i)}-\sum_{i=0}^{M}u_{i}||_{k,p(V)}\leq\sum_{i=0}^{M}||u_{i}^{\epsilon(\delta,i)}-u_{i}||_{k,p(V)}=
\]
...Teď máme normy funkcí, co mají nenulový nosič pouze na $U_{i}$:

\[
=\sum_{i=0}^{M}(||u_{i}^{\epsilon(\delta,i)}-u_{i}||_{k,p(U_{i})})\leq\sum_{i=0}^{M}\frac{\delta}{2^{i+1}}\leq\delta
\]
...Přesně dle volby $\epsilon(\delta,i)$ a zvětšením na součet nekonečné
sumy.


\subsection*{\pagebreak{}}


\subsection*{Globální aproximace s $C^{\infty}(\bar{\Omega})$}

Pro $p\in[1,\infty),\,\Omega\in C^{0,0}$ otevřená:

$\forall u\in W^{k,p}(\Omega)\,\exists u^{n}\in C^{\infty}(\bar{\Omega})$,
že posloupnost $u^{n}\longrightarrow u$ konverguje ve $W^{k,p}(\Omega)$.


\paragraph*{D}

Z definice $C^{0,0}$ máme krabice $\Omega_{r}=\Omega_{r}^{+}\cup\Omega_{r}^{-}\cup\delta_{n}\Omega$,
$r=1...M$, otevřené, pokrývají okolí hranice. Dodefinujeme $\Omega_{0}\subset\subset\Omega$
otevřenou, aby platilo $\bigcup_{i=0}^{M}\Omega_{i}\supseteq\Omega$.
Mám pokrytou $\mbox{\ensuremath{\Omega}}$ otevřenými množinami, najdu
rozklad jedničky - $u_{i}:=u\xi_{i}$, pak $u(x):=\sum u_{i}(x)$.
Proto $||u_{i}||_{k,p(\Omega_{i})}\leq||u||_{k,p(\Omega)}$ a tedy
stačí aproximovat jen $u_{i}$ - chceme najít $u_{i}^{n}\in C^{\infty}(\bar{\Omega})$,
$u_{i}^{n}\rightarrow u_{i}$ ve $W^{k,p}(\Omega)$.


\paragraph*{Na \textmd{$\Omega_{0}$:}}

Najdeme $\epsilon_{0}=dist(\delta\Omega_{0},\delta\Omega)$ a $\forall\epsilon<\epsilon_{0}:\, u_{0}^{\epsilon}:=\rho_{\epsilon}*u_{0}\in C^{\infty}(\bar{\Omega})$,
potom $||u_{0}^{\epsilon}-u_{0}||_{k,p(\Omega_{i})}\rightarrow0$.


\paragraph*{Na \textmd{$\Omega_{i>0}$:}}

Budeme konvolutit posunutou funkci! $u_{i}^{\delta}(x):=u_{i}(x',x_{d}+\delta)$.
Posunuli jsme jí aby přečnívala mimo $\Omega$ (Značíme $x=(x_{1}...x_{d})=(x',x_{d})$
jako v definici.)

Hledaná posloupnost je $u_{i}^{\epsilon}:=\rho_{\epsilon}*u_{i}^{\delta}\in C^{\infty}(\bar{\Omega})$,
z hlediska limity je také nutno říct, že volíme $\epsilon\sim\delta$.

Platí požadované $||u_{i}^{\epsilon}-u_{0}||_{k,p(\Omega)}\rightarrow0$,
protože 
\begin{itemize}
\item $||u_{i}^{\delta}-u_{i}||_{k,p(\Omega)}\overset{\delta\rightarrow0_{+}}{\rightarrow}0$
...protože je to jen posunutí, které se pro $\delta=0$ rovná původní
funkci a
\item $||\rho_{\epsilon}*u_{i}^{\delta}-u_{i}^{\delta}||_{k,p(\Omega)}\overset{\epsilon\rightarrow0_{+}}{\rightarrow}0$
...což je vlastnost zhlazení (od jistého $\epsilon_{0}$, $\epsilon\leq\epsilon_{0}$).
\end{itemize}
Dohromady tedy trojúhelníková nerovnost dává:

\[
||u_{i}^{\epsilon}-u_{i}||_{k,p(\Omega)}\leq||u_{i}^{\delta}-u_{i}||_{k,p(\Omega)}+||\rho_{\epsilon}*u_{i}^{\delta}-u_{i}^{\delta}||_{k,p(\Omega)}\rightarrow0
\]



\paragraph*{Co by se pokazilo}

kdyby nebyl splněn nějaký předpoklad? TODOD


\subsection*{\pagebreak{}}


\subsection*{Charakterizace soboleva diferenčními kvocienty (characterization
of Sobolev spaces with the help of dierence quotients)}

Pro $\Omega\subseteq\mathbb{R}^{d}$ otevřenou a $u\in L^{p}(\Omega)$
definujeme (``seminormu na $W^{1,p}$''):

\[
|u|_{p}:=\sup_{V\subset\subset\Omega}\,\sup_{i=1...d}\,\sup_{0<h<1/4\mathrm{dist}(V,\delta\Omega)}||\frac{u(x+he_{i})-u(x)}{h}||_{L^{p}(x\in V)}
\]


Platí věta pro $\Omega\subseteq\mathbb{R}^{d}$ otevřená a omezená
- pak:
\begin{enumerate}
\item Pokud $p\in[1,\infty)$, pak $u\in W^{1,p}(\Omega)\Rightarrow|u|_{p}<\infty$,
dokonce $|u|_{p}\leq c||u||_{1,p,\Omega}$.
\item Pokud $p\in(1,\infty]$, pak $|u|_{p}<\infty\Rightarrow u\in W^{1,p}(\Omega)$,
dokonce $||u||_{1,p,\Omega}\leq c(|u|_{p}+||u||_{p})$.
\end{enumerate}
{\footnotesize{Zapamatovat si $p<\infty$ u }}$\leq c||u||_{1,p,\Omega}${\footnotesize{,
ptže to vyžadují aproximační věty.}}{\footnotesize \par}


\paragraph*{Důkaz}

1. Dokážeme na lib. fixním $V\subset\subset\Omega$ nerovnost $\int_{V}\frac{u(x+he_{i})-u(x)}{h}dx\leq c||u||_{1,p,\Omega}$
pro aproximující posloupnost hladkých funkcí, nerovnost bude zachovaná
pro limitní přechod. Aproximujeme, protože pro hladké umíme dělat
operace, co je potřeba v důkazu. Proto je $p<\infty$, vyžadují to
aproximační věty.

Rozepíšeme hladkou funkci jako integrál z derivace (výraz má smysl,
protože $h$ má omezenou vzdálenost k hranici): 

\[
|u^{n}(x+he_{i})-u^{n}(x)|=|\int_{0}^{h}\frac{d}{dt}u^{n}(x+te_{i})\, dt|=
\]


...zderivováním, trojůhelníkovou nerovností a zvětšením na normu gradientu:

\[
=|\int_{0}^{h}\frac{\partial u^{n}(x+te_{i})}{\partial x_{i}}\, dt|\leq\int_{0}^{h}|\nabla u^{n}(x+te_{i})|\cdot1\, dt\leq
\]


...Holderovou nerovností (jednička je z $p'$, gradient z předpokladu
z $p$):

\[
\leq(\int_{0}^{h}|\nabla u^{n}(x+te_{i})|^{p}\, dt)^{1/p}\cdot h^{1/p'}
\]


Díky tomu se $h$ v čitateli diferenčního kvocientu zkrátí takto ($p/p'-p=-1$):

\[
|\frac{u^{n}(x+he_{i})-u^{n}(x)}{h}|^{p}\leq\frac{h^{p/p'}}{h^{p}}\int_{0}^{h}|\nabla u^{n}(x+te_{i})|^{p}\, dt=\frac{1}{h}\int_{0}^{h}|\nabla u^{n}(x+te_{i})|^{p}\, dt
\]


A konečně lze tedy odhadnout přímo integrál z diferenčního kvocientu:

\[
\int_{V}|\frac{u^{n}(x+he_{i})-u^{n}(x)}{h}|^{p}\leq\frac{1}{h}\int_{V}\int_{0}^{h}|\nabla u^{n}(x+te_{i})|^{p}\, dt\, dx=
\]


...záměnou integrálů a zvětšním na celé omega (lze, protože funkce
má kompaktní nosič ve $V$) dostaneme odhad normy gradientu v $L^{p}$
(který je odhadlý normou):

\[
=\frac{1}{h}\int_{0}^{h}\int_{V}|\nabla u^{n}(x+te_{i})|^{p}\, dx\, dt\leq\frac{1}{h}\int_{0}^{h}\int_{\Omega}|\nabla u^{n}(x)|^{p}\, dx\, dt\leq||\nabla u^{n}||_{p}^{p}\,(\leq||u^{n}||_{1,p}^{p})
\]


Dokázali jsme potřebnou nerovnost pro každý člen posloupnosti aproximujících
funkcí, tedy limitou se nerovnost přenese na limitní funkci.

2. Víme $||\frac{u(x+he_{i})-u(x)}{h}||_{L^{p}(x\in V)}\leq|u|_{p}$
(pro všechny $V,h,i$), označíme $v_{i}^{h}:=\frac{u(x+he_{i})-u(x)}{h}$,
budeme chtít ukázat, že $v_{i}^{h}$ má limitu a že ta limita se rovná
parciálním derivacím $u$, pak budou tyto parciální derivace omezené
a bude platit znění věty.

Pro $1<p<\infty$ máme z reflexivity slabě konvergující podposloupnost,
protože $v$ je omezená:

\[
v_{i}^{h_{n}}\overset{h_{n}\rightarrow0}{\rightharpoonup}v_{i}\textrm{ v }L^{p}(V)
\]


Pro $p=\infty$ máme jen $w*$ konvergující podposloupnost:

\[
v_{i}^{h_{n}}\overset{h_{n}\rightarrow0}{\rightharpoonup^{\star}}v_{i}\textrm{ v }L^{\infty}(V)
\]


V obou případech ale platí:

\[
||v_{i}||_{p}\leq\liminf_{h_{n}\rightarrow0}||v_{i}^{h_{n}}||_{p}\leq|u|_{p}
\]


Když dokážeme, že $v_{i}=\frac{\partial u}{\partial x_{i}}$ ve $V$,
budeme hotovi. Ukážeme že je přímo slabou derivací, protože díky slabým
konvergencím budeme moct přejít k limitám v integrálech (integrál
tu hraje roli lineárního funkcionálu pro slabou konvergenci):

\[
\forall\varphi\in\mathcal{D}(V):\,\int_{V}v_{i}\varphi\, dx=\lim_{h\rightarrow0}\int_{V}v_{i}^{h}(x)\varphi(x)\, dx=
\]


...lze přejít k integrálu přez $\mathbb{R}^{d}$, protože $\varphi$
má kompaktní support pouze ve $V$. Děláme to proto, že chceme přehodit
derivační podíl z $u$ na $\varphi$ a to ve dvou krocích - nejprve
substitucí (s jednotkovým jakobiánem a neměnící nekonečné meze) $x_{new}=x_{old}-he_{i}$:

\[
=\lim_{h\rightarrow0}\int_{\mathbb{R}^{d}}\frac{u(x+he_{i})-u(x)}{h}\varphi(x)\, dx=\lim_{h\rightarrow0}\int_{\mathbb{R}^{d}}\frac{u(x)-u(x-he_{i})}{h}\varphi(x-he_{i})\, dx=
\]


... druhým krokem bude rozdělení (aritmetikou limit a integrálů) na
dva integrály, zpětnou substitucí na jednom z nich a zpětném sečtení:

\[
=\lim_{h\rightarrow0}\frac{1}{h}\int_{\mathbb{R}^{d}}u(x)\varphi(x-he_{i})-u(x-he_{i})\varphi(x-he_{i})\, dx=\lim_{h\rightarrow0}\frac{1}{h}\int_{\mathbb{R}^{d}}u(x)\varphi(x-he_{i})-u(x)\varphi(x)\, dx=
\]


...což vyústilo ve faktické prohození role $u$ a $\varphi$ (bez
přechodu k $\mathbb{R}^{d}$ by to mimochodem nebylo možné):

\[
=\lim_{h\rightarrow0}\int_{\mathbb{R}^{d}}\frac{\varphi(x-he_{i})-\varphi(x)}{h}u(x)\, dx=
\]


...nakonec zaměníme limitu a integrál -tím získáme derivaci- a přejdeme
zpět k oblasti $V$ (což stále lze, protože mimo ní je vše nulové):

\[
=-\int_{\Omega}\lim_{h\rightarrow0}\frac{\varphi(x-he_{i})-\varphi(x)}{h}u(x)\, dx=-\int_{\Omega}\frac{\partial\varphi(x)}{\partial x_{i}}u(x)\, dx
\]


Hotovo, ověřený vztah pro slabou derivaci (platí na každé $V\subset\subset\Omega$).

\pagebreak{}


\subsection*{Extension Theorem}

Nechť $\Omega\subseteq\mathbb{R}^{d}$ otevřená omezená $C^{0,1}$
lipschitzovská. Pak existuje $\Omega\subseteq\Omega'\subseteq\mathbb{R}^{d}$,
že $\forall p\in[1,\infty)\,\exists E:\, W^{1,p}(\Omega)\rightarrow W^{1,p}(\mathbb{R}^{d})$,
které je lineární, omezené spojité, že $Eu(x)=u(x)$ s.v v $\Omega$
a $Eu(x)=0$ s.v. v $\mathbb{R}^{d}\backslash\Omega'$.


\subsection*{\pagebreak{}}


\subsection*{Gagliardo-Nirenberg inequality}


\paragraph*{
\[
\forall u\in\mathcal{D}(\mathbb{R}^{d}):\,||u||_{L^{\frac{d}{d-1}}(\mathbb{R}^{d})}\leq||\triangledown u||_{L^{1}(\mathbb{R}^{d})}
\]
D}

Protože je $u$ hladká funkce s kompaktním supportem:

\[
\forall i=1...d:\, u(x)=\int_{-\infty}^{x_{i}}\frac{\partial u}{\partial s}(x_{1}...x_{i-1},s,x_{i+1}...x_{d})\, ds
\]


\[
|u(x)|\leq|\int_{-\infty}^{x_{i}}\frac{\partial u}{\partial s}(x_{1}...x_{i-1},s,x_{i+1}...x_{d})\, ds|\leq\int_{-\infty}^{x_{i}}|\nabla u(x)|
\]


\[
U_{i}(x_{1}...x_{i-1},x_{i+1}...x_{d}):=\int_{-\infty}^{\infty}|\nabla u(x)|\, dx_{i}...\Rightarrow|u(x)|\leq U_{i}(x_{1}...x_{d})\,(\forall i)
\]


\[
\Rightarrow|u(x)|^{1/(d-1)}\leq U_{i}^{1/(d-1)}\Rightarrow|u(x)|^{d/(d-1)}\leq\prod_{i=1}^{d}U_{i}^{1/(d-1)}
\]


...zintegrujeme nerovnost $dx_{1}$:

\[
\int_{-\infty}^{\infty}|u(x)|^{d/(d-1)}\, dx_{1}\leq\int_{-\infty}^{\infty}\prod_{i=1}^{d}U_{i}^{1/(d-1)}\, dx_{1}=
\]


...a protože $U_{1}$ nezávisí na $x_{1}$:

\[
=U_{1}^{1/(d-1)}\int_{-\infty}^{\infty}\prod_{i=2}^{d}U_{i}^{1/(d-1)}\, dx_{1}\leq
\]


...zavedeme $p_{i}$, $i=2...d$, aby $\sum_{i=2}^{d}1/p_{i}=1$,
potom Holderem:

\[
\leq U_{1}^{1/(d-1)}\prod_{i=2}^{d}(\int_{-\infty}^{\infty}U_{i}^{p_{i}/(d-1)}\, dx_{1})^{1/p_{i}}=
\]


...zvolíme $p_{i}=d-1$:

\[
=U_{1}^{1/(d-1)}\prod_{i=2}^{d}(\int_{-\infty}^{\infty}U_{i}\, dx_{1})^{1/(d-1)}
\]


...dostali jsme novou nerovnost, znovu zintegrujeme dle $x_{2}$:

\[
\int_{-\infty}^{\infty}\int_{-\infty}^{\infty}|u(x)|^{d/(d-1)}\, dx_{1}\, dx_{2}\leq\int_{-\infty}^{\infty}U_{1}^{1/(d-1)}\prod_{i=2}^{d}(\int_{-\infty}^{\infty}U_{i}\, dx_{1})^{1/(d-1)}\, dx_{2}=
\]


...$U_{2}$ nezávisí na $x_{2}$:

\[
=(\int_{-\infty}^{\infty}U_{2}\, dx_{1})^{1/(d-1)}\int_{-\infty}^{\infty}U_{1}^{1/(d-1)}\prod_{i=3}^{d}(\int_{-\infty}^{\infty}U_{i}\, dx_{1})^{1/(d-1)}\, dx_{2}\leq
\]


...a zase Holdera na $d-1$ členů (ano, i na člen s $U_{1}$ a ne
na člen s $U_{2}$):

\[
\leq(\int_{-\infty}^{\infty}U_{2}\, dx_{1})^{1/(d-1)}(\int_{-\infty}^{\infty}U_{1}\, dx_{2})^{1/(d-1)}\prod_{i=3}^{d}(\int_{-\infty}^{\infty}U_{i}\, dx_{1}\, dx_{2})^{1/(d-1)}
\]


Opakováním, indukcí:

\[
\int_{\mathbb{R}^{d}}|u(x)|^{d/(d-1)}\, dx\leq\prod_{i=1}^{d}(\int_{\mathbb{R}^{d-1}}U_{i}dx_{1}...dx_{i-1}dx_{i+1}...dx_{d})^{1/(d-1)}=
\]


...protože jde o součin těch samých čísel:

\[
=\prod_{i=1}^{d}(\int_{\mathbb{R}^{d-1}}\int_{\mathbb{R}}|\nabla u(x)|\, dx)^{1/(d-1)}=||\nabla u||_{L^{1}(\mathbb{R}^{d})}^{d/(d-1)}
\]



\subsection*{\pagebreak{}}


\subsection*{Morrey lemma}

\[
\forall u\in\mathcal{D}(\mathbb{R}^{d}):\,|u(x)-u(y)|\leq\frac{c(d)}{\alpha}|x-y|^{\alpha}\sup_{B_{R}(x_{0})}\int_{B_{R}(x_{0})}\frac{|\triangledown u|}{R^{d-1+\alpha}}
\]



\paragraph*{D}

Budeme směřovat k cíli cestou tohoto odhadu:

\[
|u(x)-u(y)|\leq|u(x)-\fintop_{B_{R}(0)}u(z)\, dz|+|u(y)-\fintop_{B_{R}(0)}u(z)\, dz|
\]


Soustavu souřadnou si BÚNO volíme tak, aby úsečka $xy$ ležela na
bázovém vektoru $e_{d}$, střed $xy$ byl v počátku a $R$ voleno
jako vzdálenost bodu $y$ od počátku.

Odvodíme nerovnost pro jeden člen $|u(y)-\fintop_{B_{R}(0)}u(z)\, dz|$,
druhá bude kvůli symetrickým podmínkám stejná. Zapíšeme rozdíl pomocí
rovnosti, která platí pro spojité funkce díky primitivním funkcím:

\[
u(y)-\fintop_{B_{R}(0)}u(z)\, dz=-\int_{0}^{R}\frac{d}{dr}\fintop_{B_{r}((R-r)e_{d})}u(z)\, dz\, dr=
\]


...zbavíme se průměrového integrálu,$\kappa^{d}$ je míra jednotkové
koule v $\mathbb{R}^{d}$:

\[
=\frac{-1}{\kappa^{d}}\int_{0}^{R}\frac{d}{dr}\frac{1}{r^{d}}\int_{B_{r}((R-r)e_{d})}u(z)\, dz\, dr=
\]


...pomocí substituce $z=rx+(R-r)e_{d}\Rightarrow rx=z-(R-r)e_{d}$,
$J=r^{d}$:

\[
=\frac{-1}{\kappa^{d}}\int_{0}^{R}\frac{d}{dr}\int_{B_{1}(0)}u(rx+(R-r)e_{d})\, dx\, dr=
\]


...teď množina nezávisí na $r$, můžeme derivovat:

\[
=\frac{-1}{\kappa^{d}}\int_{0}^{R}\int_{B_{1}(0)}\frac{\partial u(rx+(R-r)e_{d})}{\partial x_{i}}(x_{i}-\delta_{id})\, dx\, dr=
\]


...teď zpět substituci. $e_{d}$ se změní na $\delta_{i,d}$, protože
uděláme operaci po složkách:

\[
=\frac{-1}{\kappa^{d}}\int_{0}^{R}\frac{1}{r^{d}}\int_{B_{r}((R-r)e_{d})}\sum_{i=1}^{d}\frac{\partial u(z)}{\partial z_{i}}(\frac{z_{i}-(R-r)\delta_{id}}{r}-\delta_{id})\, dz\, dr=
\]


\[
=\frac{-1}{\kappa^{d}}\int_{0}^{R}\frac{1}{r^{d}}\int_{B_{r}((R-r)e_{d})}\sum_{i=1}^{d}\frac{\partial u(z)}{\partial z_{i}}(\frac{z_{i}-R\delta_{id}}{r})\, dz\, dr
\]


Odhadem absolutních hodnot, sečtením parciálních derivací na gradient
a využitím faktu, že člen $(\frac{z_{i}-R\delta_{id}}{r})$ lze schovat
do konstanty, dostáváme nerovnost:

\[
|u(y)-\fintop_{B_{R}(0)}u(z)\, dz|\leq2c(d)\int_{0}^{R}\frac{r^{d-1+\alpha}}{r^{d}}\int_{B_{r}((R-r)e_{d})}\frac{|\nabla u(z)|}{r^{d-1+\alpha}}\, dz\, dr\leq
\]


...kde stačí přejít k suprému integrandu (suprému přez všechny koule
a všechny počátky), čímž se stane z hlediska integrálu $dr$ konstantou
a lze vytknout mimo tento integrál:

\[
\leq2c(d)\int_{0}^{R}r^{\alpha-1}\, dr\sup_{B_{R}(x_{0})}\int_{B_{R}(x_{0})}\frac{|\triangledown u(z)|}{R^{d-1+\alpha}}\, dz=\frac{2c(d)}{\alpha}R^{\alpha}\sup_{B_{R}(x_{0})}\int_{B_{R}(x_{0})}\frac{|\triangledown u(z)|}{R^{d-1+\alpha}}
\]


Výslednou nerovnost aplikujeme na oba dva členy a to tak, že posuneme
problém aby $x$ a $y$ bylo vystředěné do počátku soustavy souřadné,
kde použijeme společný poloměr $R=\frac{|x-y|}{2}$. (Je to jednodušší
postup, než se v nerovnostech a odvození nahoře tahat s jiným počátkem
než nulou.) Dostaneme:

\[
|u(x)-u(y)|\leq|u(x)-\fintop_{B_{R'}(0)}u(z)dz|+|u(y)-\fintop_{B_{R}(0)}u(z)dz|\leq\frac{4c(d)}{\alpha}R^{\alpha}S\leq\frac{c(d)}{\alpha}S|x-y|^{\alpha}
\]



\subsection*{\pagebreak{}}


\subsection*{Continuous and compact embedding of Sobolev spaces into Lebesgue
spaces or into Hoder spaces}


\paragraph*{Spojité vnoření}
\begin{lyxcode}
$X\hookrightarrow Y\Leftrightarrow X\subset Y\,\&\&\,\exists c>0:\,\forall u\in X\,||u||_{Y}\leq c||u||_{X}$
\end{lyxcode}

\paragraph*{Kompaktní vnoření}

$X\hookrightarrow\hookrightarrow M$ banachových prostorů: $\{x_{n}\}_{n=1}^{\infty},||x_{n}||_{X}\leq c\Rightarrow\exists\{u_{m}\}_{m=1}^{\infty}\subseteq\{x_{n}\}_{n=1}^{\infty}:\, u_{n}\rightarrow u$
v $M$.


\paragraph*{Věta vnoření}

Nechť $\Omega\in C^{0,1}$ (Lipschitzovská):

\[
W^{1,p}(\Omega)\hookrightarrow L^{p*}(\Omega),\, p^{\star}=\frac{dp}{d-p}\,...\, p<d
\]


\[
W^{1,p}(\Omega)\hookrightarrow L^{p*}(\Omega),\, p^{\star}\in(1,\infty)\,...\, p=d
\]


\[
W^{1,p}(\Omega)\hookrightarrow C^{0,\alpha}(\bar{\Omega}),\,\alpha=1-d/p\,...\, p>d
\]


Pokud bychom chtěli $\Omega=\mathbb{R}^{d}$, musíme se v uvedených
vztazích omezit na prostor $W_{0}^{1,p}(\mathbb{R}^{d})$.


\subparagraph*{Navíc platí Kompaktní vnoření pro ostré nerovnosti:}

\[
W^{1,p}(\Omega)\hookrightarrow\hookrightarrow L^{q}(\Omega),\,\forall q<p^{\star}
\]


\[
W^{1,p}(\Omega)\hookrightarrow C^{0,\beta}(\bar{\Omega}),\,\forall0\leq\beta<\alpha=1-d/p
\]


{\footnotesize{Pozor na $\bar{\Omega}$ a Lipschitzovskost kvůli rozšíření.}}{\footnotesize \par}


\paragraph*{Důkaz}

Protože je $\Omega$ lipschitzovská, lze každá funkce $u$ z $W^{1,p}(\Omega)$
rozšířit na $\mathbb{R}^{d}$ (se zachováním normy až na násobek konstanty)
aby měla rozšířená $Eu$ kompaktní nosič v $\mathbb{R}^{d}$. V takovém
prostoru jsou husté hladké funkce s kompaktním nosičem, takže stačí
jen pro tyto funkce dokázat následující nerovnost, která omezuje velikost
norem v cílovém prostoru vnoření. Protože lze tato nerovnost dokázat
pro funkce s kompaktním nosičem $\forall u\in\mathcal{D}(\mathbb{R}^{d})$,
platí vnoření i pro $W_{0}^{1,p}(\mathbb{R}^{d})$. Chci dokázat:

\[
\exists c(p,d),\,||u||_{L^{p^{\star}}(\mathbb{R}^{d})}\leq c(p,d)||\nabla u||_{L^{p}(\mathbb{R}^{d})}\forall u\in\mathcal{D}(\mathbb{R}^{d})
\]


Pokud je $p=1$, je to přímo Gagliardo Niranberg, pro větší $p$:

\[
\int_{\mathbb{R}^{d}}|u|^{p^{\star}}dx=\int_{\mathbb{R}^{d}}\left(|u|^{\frac{p^{\star}(d-1)}{d}}\right)^{\frac{d}{d-1}}dx=||\,|u|^{\frac{p^{\star}(d-1)}{d}}||_{\frac{d}{d-1}}^{d/(d-1)}\leq
\]


...dle Gagliardo Nirenberga máme teď ve tvaru kdy lze použít odhad:

\[
\leq\left(\int_{\mathbb{R}^{d}}|\,\nabla|u|^{\frac{p^{\star}(d-1)}{d}}|dx\right)^{\frac{d}{d-1}}\leq
\]


...derivujeme slouženou funkci $\nabla|u|$ umocněnou. Konstanta vyskočí
derivováním mocniny $c(p,d)=\frac{p^{\star}(d-1)}{d}$:

\[
\leq c(p,d)\left(\int_{\mathbb{R}^{d}}|u|^{\frac{p^{\star}(d-1)}{d}-1}|\nabla u|dx\right)^{\frac{d}{d-1}}\leq
\]


...Holderova nerovnost! První násobitel v integrálu je $L^{p'}$ (duál),
druhý je $L^{p}$:

\[
\leq c(p,d)||\nabla u||_{p}^{d/(d-1)}\left(\int_{\mathbb{R}^{d}}|u|^{\frac{p^{\star}(d-1)-d}{d}\frac{p}{p-1}}dx\right)^{\frac{p-1}{p}\frac{d}{d-1}}
\]


Dostali jsme jistou nerovnost, kterou upravíme - jednak z levé strany
uděláme normu v $L^{p^{\star}}$ prostoru odmocněním obou stran a
pravou stranu také upravíme na normu $L^{p^{\star}}$, protože $\frac{p^{\star}(d-1)-d}{d}\frac{p}{p-1}=p^{\star}$.

Druhý koeficient lze také upravit do tvaru, ve kterém bude lépe vidět
co se stane v následujícím kroku $\frac{p-1}{p}\frac{d}{d-1}\frac{1}{p^{\star}}=1-\frac{d}{(d-1)p^{\star}}$
(je to přesně $1$ mínus koeficient u durhého členu pravé strany)...

\[
||u||_{L^{p^{\star}}}\leq c(p,d)||\nabla u||_{p}^{\frac{1}{p^{\star}}\frac{d}{d-1}}||u||_{L^{p^{\star}}}^{1-\frac{d}{(d-1)p^{\star}}}
\]


...protože teď rovnost vydělíme, převedeme normu $L^{p^{\star}}$
na levou stranu a odmocníme:

\[
||u||_{L^{p^{\star}}}\leq c(p,d)||\nabla u||_{p}
\]



\paragraph*{Případ $C^{0,\alpha}$}

Dle Morayova lemmatu:

\[
\frac{|u(x)-u(y)|}{|x-y|^{\alpha}}\leq\frac{c(d)}{\alpha}S=\frac{c(d)}{\alpha}\sup_{B_{R}(x_{0})}\int_{B_{R}(x_{0})}\frac{|\triangledown u|\cdot1}{R^{d-1+\alpha}}dx\leq
\]


...použitím Holderovy nerovnosti na integrál (proto jsme si tam připravili
$1$, ta bude z $L^{p'}$, gradient z $L^{p}$):

\[
\leq\frac{c(d)}{\alpha}\sup_{B_{R}(x_{0})}\frac{1}{R^{d-1+\alpha}}\left(\int_{B_{R}(x_{0})}|\triangledown u|\, dx\right)^{1/p}\cdot|B_{R}(x_{0})|^{1/p'}\leq
\]


...upravíme oba členy: člen $B_{R}(x_{0})$ je roven objemu jednotkové
koule (vytkneme ze supréma do konstanty) krát zvětšení dané objemem
v prostoru $R^{\frac{d}{p'}}$ (necháme v suprému), člen s gradientem
zvětšíme na normu $W^{1,p}$ přez celý prostor - tím se zbaví závislosti
na $R$ a lze dát pryč ze supréma. Co zbyde:

\[
\leq\frac{c(d,\alpha)}{\alpha}||u||_{1,p}\sup_{R}\frac{R^{\frac{d}{p'}}}{R^{d-1+\alpha}}\leq
\]


...vzpomeneme, že $p'=p/(p-1)$ a $\alpha=1-d/p$ a dosadíme. Vyjde
přesně exponent nula, tedy suprémum má vždy hodnotu jedna:

\[
\leq\frac{c(d,\alpha)}{\alpha}||u||_{1,p}\sup_{R}R^{\frac{d(p-1)}{p}-(d-1+1-d/p)}=\frac{c(d,\alpha)}{\alpha}||u||_{1,p}
\]


Tedy Holderovská norma je kontrolovaná Sobolevovskou normou, což dokazuje
spojité vnoření.


\paragraph*{Proč kompaktní?}

Případ $C^{0,\alpha}$ - protože Arzela-Ascoli: $C^{0,\alpha}\hookrightarrow\hookrightarrow C^{0,\beta}$
pro $\alpha>\beta$.


\paragraph*{Případ $L^{q}$:}


\subparagraph*{Krok 1}

Chceme dokázat $W^{1,1}(\Omega)\hookrightarrow\hookrightarrow L^{1}(\Omega)$.

Víme $W^{1,1}\hookrightarrow L^{d/(d-1)}$, což je reflexivní prostor,
tedy pro zadanou $u^{n}\in W^{1,p}(\Omega),\,||u^{n}||_{1,1}\leq c$
máme ze spojitého vnoření $||u^{n}||_{L^{d/(d-1)}}\leq c$, kde zas
máme z reflexivity $u^{n}\rightharpoonup u\textrm{ v }L^{d/(d-1)}(\Omega)$
(búno s přechodem k podposloupnosti). Protože je $\Omega$ omezená,
konverguje $u^{n}$ slabě i v $L^{1}$.

Teď ukážeme, že $||u^{n}-u||_{1}\leq\epsilon$, což potvrdí kompaktní
vnoření. Ukážeme to odhadnutím této normy pro fixní libovolné $\epsilon>0$:

\[
||u^{n}-u||_{1}\leq||u^{n}-\rho_{\epsilon}*u^{n}||_{1}+||\rho_{\epsilon}*(u^{n}-u)||_{1}+||u-\rho_{\epsilon}*u||_{1}
\]


...třetí člen upravíme do podoby limity díky slabé zdola polospojitosti
normy $||u-\rho_{\epsilon}*u||_{1}\leq\lim||u^{n}-\rho_{\epsilon}*u^{n}||_{1}$:

\[
||u^{n}-u||_{1}\leq2||u^{n}-\rho_{\epsilon}*u^{n}||_{1}+||\rho_{\epsilon}*(u^{n}-u)||_{1}
\]


První člen ($u^{n}-\rho_{\epsilon}*u^{n}=\int_{\mathbb{R}^{d}}\rho_{\epsilon}(x-y)(u^{n}(x)-u^{n}(y))\, dy$).
Pro jeho odhad v $L^{1}$ normě platí: 
\[
||u^{n}-\rho_{\epsilon}*u^{n}||_{1}\leq\int_{\mathbb{R}^{d}}\int_{\mathbb{R}^{d}}\rho_{\epsilon}(x-y)(u^{n}(x)-u^{n}(y))\, dy\, dx=
\]


...substitucí $x-y=z$:

\[
=\int_{\mathbb{R}^{d}}\int_{\mathbb{R}^{d}}|z|\rho_{\epsilon}(z)\frac{(u^{n}(y+z)-u^{n}(y))}{|z|}\, dy\, dz\leq
\]


...zlomek lze odhadnout $W^{1,1}$ normou (věta o diferenčních kvocientech):

\[
=||u^{n}||_{1,1}\int_{\mathbb{R}^{d}}|z|\rho_{\epsilon}(z)\, dz\leq
\]


...protože zmenšováním $\epsilon$ se zmenšuje i nosič $|z|$, který
lze odhadnout přímo $\epsilon$, zatímco tam zbyde integrál ze zhlazovací
funkce, který je roven jedné, lze integrál vzít libovolně malý:

\[
\leq||u^{n}||_{1,1}\epsilon\int_{\mathbb{R}^{d}}\rho_{\epsilon}(z)\, dz\leq\epsilon||u^{n}||_{1,1}
\]


Druhý člen ($\rho_{\epsilon}*(u^{n}-u)=\int_{\mathbb{R}^{d}}\rho_{\epsilon}(x-y)(u^{n}(x)-u(y))\, dy$).

Protože $u^{n}-u$ jde slabě k nule a násobí se jen omezenou funkcí,
tak celek jde k nule. 

Tedy celkově $||u^{n}-u||_{1}\rightarrow0$, protože pravá strana
nerovnosti jde k nule.


\subparagraph*{Krok 2}

Máme $W^{1,p}(\Omega)\hookrightarrow L^{p^{\star}}(\Omega)\,\&\&\, W^{1,1}(\Omega)\hookrightarrow\hookrightarrow L^{1}(\Omega)$.
Vezměme posloupnost $u^{n}\in W^{1,p}(\Omega),\,||u^{n}||_{1,p}\leq c$.

Díky tomu platí:
\begin{itemize}
\item $||u^{n}||_{1,1}\leq c$ ... protože je $\Omega$ omezená
\item $||u^{n}||_{L^{p^{\star}}}\leq c$ ... ze spojitého vnoření
\item máme i podposloupnost $u^{m}\subseteq u^{n},\, u^{m}\rightharpoonup u\textrm{ v }L^{p^{\star}}$
(slabě), protože máme reflexivní prostor.
\item $u^{m}\rightarrow u\textrm{ v }L^{1}$ ... kvůli kompaktnímu vnoření
\end{itemize}
Teď lze použít interpolační nerovnost pro důkaz kompaktního vnoření:


\subparagraph*{Interpolační nerovnost}

\[
||u||_{L^{q}}\leq||u||_{L^{p}}^{\alpha}\,||u||_{L^{r}}^{1-\alpha}\textrm{ pro }p\leq q\leq r\textrm{, kde }1/q=\alpha/p+(1-\alpha)/r
\]


{\footnotesize{(Dokazuje se $\int|u|^{q}=\int|u|^{q\gamma}|u|^{(1-\gamma)q}$
a pak Holdera)}}{\footnotesize \par}


\paragraph*{Aplikujeme }

ji na náš případ, tzn $p=1\leq q\leq r=p^{\star}$:

\[
||u^{n}-u||_{L^{q}}\leq||u^{n}-u||_{L^{1}}^{\alpha}\,||u^{n}-u||_{L^{p^{\star}}}^{1-\alpha}
\]


Na pravé straně máme díky odvozeným konvergenčním a slabě konvergenčním
vztahům součin nulová krát omezená (v tomto pořadí), tedy levá strana
jde k nule, což je tvrzení silné konvergence z kompaktního vnoření.

\begin{comment}


\paragraph*{Komentář}

Umím to pro $\mathbb{R}^{d}$, platí to i tam, kde hladké fce $u$
jsou hladké, například v W01,1. Obecně pro W01,1(Omega) ne - musím
extension theorem. Pokud neni 0 ve W01,1, tak tam uz fce nejsou husté,
proto chci aby omega byla C0, pak uz kazda sobolevovska fubkce lze
natahnout na cele Rn a normy jsou ekvivalentni.

Ten odhad co mam pouziju na rozsirenou funkci ktera je nula mimo Omega0.

\end{comment}


\subsection*{\pagebreak{}}


\subsection*{Trace theorem}

Nechť $\Omega\in C^{0,1}$ (lipschitzovská), pak existuje spojitý
lineární operátor $T$: $W^{1,p}(\Omega)\rightarrow L^{q}(\delta\Omega)$,
(resp. do $C(\delta\Omega)$), takový že

$\forall u\in W^{1,p}(\Omega)\cap C^{\infty}(\bar{\Omega}):\textrm{ }Tu=u|_{\delta\Omega}$

Pro $p\in[1,d)$ je $q\leq p^{\#}$, $p^{\#}=\frac{dp-p}{d-p}$.

Pro $p=d$ je $q\in[1,\infty)$ libovolné

Pro $p>d$ je $q\in[1,\infty)$ libovolné, dokonce máme vnoření $W^{1,p}(\Omega)\hookrightarrow C^{0,\alpha}(\bar{\Omega})$.


\paragraph*{D}


\paragraph*{BÚNA}

Případ $p>d$ je jasný dokonce z vnoření do spojitých funkcí. Případ
$p=d$ má stejný důkaz jako $p<d$, stačí tedy dokázat jen tento.

BÚNO stačí uvažovat $q=p^{\#}$, protože z předpokladu konečnosti
hranice máme vnoření prostorů do sebe.


\paragraph*{Jak operátor zadefinujeme}

Zadanou funkci můžeme větou Globální aproximace až do hranice aproximovat
pomocí $u_{n}\rightarrow u\textrm{, }u_{n}\in W^{1,p}(\Omega)\cap C^{\infty}(\bar{\Omega})$.
Kdybychom uměli ukázat Cauchyovskost posloupnosti $u_{n}|_{\delta\Omega}$
v prostoru $L^{p^{\#}}(\delta\Omega)$, měla by posloupnost $u_{n}|_{\delta\Omega}$
limitu a tu bychom zadefinovali jako výsledek působení operátoru na
zadanou funkci $Tu:=\lim_{n\rightarrow\infty}u_{n}|_{\delta\Omega}$.

Z cauchyovskosti by pak také plynulo, že je nezávislá na aproximující
posloupnosti.

Cauchyovskost v prostoru $L^{p^{\#}}(\delta\Omega)$ dostaneme, pokud
ukážeme nerovnost (ta přenese cauchyovskost $u_{n}$ na $u_{n}|_{\delta\Omega}$):

\[
\forall v\in C^{\infty}(\bar{\Omega})\textrm{: }||v||_{L^{p^{\#}}(\delta\Omega)}\leq c(p,\Omega)||v||_{W^{1,p}(\Omega)}
\]



\paragraph*{Redukce na krabičky}

Díky lipschitzovskosti hranice, stačí dokázat tuhle nerovnost na libovolném
okolí hranice. Proč - z $\Omega\in C^{0,1}$ existuje totiž pokrytí
hranice, danou funkci $v$ lze rozložit rozkladem jednotky a pokud
bude tahle nerovnost platit pro libovolnou komponentu rozložené funkce,
sečte se příslušný (konečný) počet nerovností na celý prostor. 

Mějme tedy z definice (ve správných souřadnicích) okolí hranice $\Omega^{+}$,
funkci $a$, tloušťku $\beta$, šířku $\alpha$ definičního oboru
$\triangle=\{x',|x'|<\alpha\}$ funkce $a$ a zadanou funkci $v$
s nosičem kompaktně uvnitř $\Omega^{+}$ (z rozkladu jednotky):

Napíšeme si funkci proměnné $x'=(x_{1}...x_{d-1})$, která je hladká
a tedy lze zapsat jako integrál z derivace:

\[
|u(x',a(x'))|^{\frac{dp-p}{d-p}}=-\int_{a(x')}^{a(x')+\beta}\frac{\partial}{\partial\eta}\left(|u(x',\eta)|^{\frac{dp-p}{d-p}}\right)d\eta
\]


...dalším rozderivováním (konstanta bude koeficient z derivování mociny,
všimněme si i změny exponentu odečtením jedničky) a zvětšením $|\partial u/\partial\eta|\leq|\nabla u|$:

\[
|u(x',a(x'))|^{\frac{dp-p}{d-p}}\leq c(p,d)\int_{a(x')}^{a(x')+\beta}|\nabla u(x',\eta)|\,|u(x',\eta)|^{\frac{dp-d}{d-p}}d\eta
\]


...zintegrujeme přez $\triangle$, tím dostaneme integraci přez celou
d-dimenzionální oblast (díky už existujícímu integrálu $d\eta$) a
použijeme Holderovu nerovnost (TODOD?? jak???): 

\[
\int_{\triangle}|u(x',a(x'))|^{\frac{dp-p}{d-p}}\leq c(p,d)||\nabla u(x',\eta)||_{L^{p}(\Omega^{+})}||u||_{L^{\frac{dp}{d-p}}(\Omega^{+})}^{\frac{d(p-1)}{d-p}}d\eta
\]


...pravou stranu upravíme na normu $W^{1,p}$. Toho docílíme odhadem
gradientu na pravé straně $||\nabla u||_{L^{p}(\Omega^{+})}\leq||u||_{W^{1,p}(\Omega^{+})}$
a větou o vnoření na člen $||u||_{L^{\frac{dp}{d-p}}(\Omega^{+})}^{\frac{d(p-1)}{d-p}}\leq c||u||_{W^{1,p}}^{\frac{d(p-1)}{d-p}}$
:

\[
\int_{\triangle}|u(x',a(x'))|^{\frac{dp-p}{d-p}}\leq c(p,d)||u||_{W^{1,p}(\Omega^{+})}^{1+\frac{d(p-1)}{d-p}}
\]


...levou stranu upravíme na normu $L^{...}$ , toho docílíme odmocněním
celé rovnice (kvůli normě vlevo). To se hodí i kvůli pravé straně,
exponent u $||u||_{W^{1,p}(\Omega^{+})}$ je totiž přímo roven $1+\frac{d(p-1)}{d-p}=\frac{p(d-1)}{d-1}$,
což je přesně hodnota, kterou celou rovnici odmocňujeme. Zůstane tedy
mocnina $1$:

\[
||u||_{L^{\frac{dp-p}{d-p}}(\delta\Omega^{+})}\leq c(p,d)||u||_{W^{1,p}(\Omega^{+})}
\]


Což je nerovnost kterou jsme chtěli na příslušné části hranice.


\subsection*{\pagebreak{}}


\subsection*{Linear and nonlinear Lax-Milgram lemma}


\paragraph*{Lineární LaxMilgram}

Nechť $V$ je $\mathbb{R}$-Hilbertprostor. Nechť $B$ je bilineární
forma $V\times V\rightarrow\mathbb{R}$, která je navíc
\begin{itemize}
\item V-Omezená $\exists M>0:\,\forall x,y\in V:\,|B(x,y)|\leq M||x||\,||y||$
\item V-Eliptická $\exists\alpha>0:\,\forall x\in V:\, B(x,x)\geq\alpha||x||^{2}$
\end{itemize}
Pak pro každou $F\in V^{\star}$ existuje $u\in V$, že $B(u,v)=<F,v>\,(\forall v\in V)$.

Navíc platí apriorní odhad $||u||_{V}\leq1/\alpha\,||f||_{V^{\star}}$.


\paragraph*{D}

Přiřazení $u\rightarrow B(u,.)$ dává pevnému $u$ lineární spojité
zobrazení na $V$. To je Rieszovou větou o reprezentaci reprezentováno
nějakým prvkem prostoru $V$. Tento prvek označíme $A(u)$, protože
záleží na volbě $u$. Tedy z Riesze máme $(A(u),v)=B(u,v)\,\forall v\in V$.

Když dokážeme, že operátor $A$ je lineární, omezený a na $V$, pak
umíme pro každé zadané $F\in V^{\star}$ najít Rieszem nejprve $f\in V,\,<F,v>=(f,v)\,\forall v\in V$
a pak hledané $u$ je přesně vzor při zobrazení $A$ - $Au=f$.

Linearita $A$ - dokážeme linearitu na skalárních součinech se všemi
$v\in V$, bude A lineární:

\[
(A(\alpha u_{1}+\beta u_{2}),v)=B(\alpha u_{1}+\beta u_{2},v)=\alpha B(u_{1},v)+\beta B(u_{2},v)=\alpha A(u_{1},v)+\beta A(u_{2},v)
\]


Omezenost $A$ - dostaneme použitím V-Omezenosti $B$ a vydělením
následující rovnosti:

\[
||Au||^{2}=B(u,Au)\leq M||u||\,||Au||
\]


Surjektivnost $A$ ve dvou krocích. Nejprve ukážeme, že Range $A$
je uzavřený a následně, že na range je kolmá jen $0\in V$:

Máme tedy konvergentní posloupnost prvků range $Au^{n}\rightarrow v$
a chceme ukázat, že $v\in Rng\, A\Leftrightarrow\exists u:\, Au=v$.
Dokážeme Cauchyovskost vzorů $u^{n}$ pomocí V-elipticity:

\[
\alpha||u^{n}-u^{m}||^{2}\leq B(u^{n}-u^{m},u^{n}-u^{m})=(A(u^{n}-u^{m}),u^{n}-u^{m})\leq M||u^{n}-u^{m}||\,||Au^{n}-Au^{m}||
\]


Vydělíme a máme normu omezenou normou $||Au^{n}-Au^{m}||\rightarrow0$,
ta konverguje k nule, z předpokladu.

Máme Cauchyovskost, takže existuje $\exists u:\, u^{n}\rightarrow u$.
Protože je $A$ spojitá a z předpokladu platí $Au^{n}\rightarrow v$,
máme rovnost $Au=v$, Range je tudíž uzavřený.

Poslední krok - kdyby existoval nenulový prvek $v$, který je kolmý
na celý range:

\[
\exists v\in V:\,\forall u\in V:\,0=(Au,v)
\]


Tak by šlo zvolit $u=v$ a dostali bychom spor s elipticitou:

\[
0=(Av,v)\geq\alpha||v||^{2}>0
\]


Na Range je kolmá tedy pouze vektorová nula. Což dokončuje důkaz surjektivity
$A$.


\paragraph*{Nelineární LaxMilgram}

Nechť $V$ je $\mathbb{R}$-Hilbertprostor. Nechť $B$ je spojité
zobrazení $V\times V\rightarrow\mathbb{R}$, lineární v druhé složce
(testovacích funkcích), které je navíc:
\begin{itemize}
\item Lipschitzovsky spojité (v 1. složce) $\exists M>0:\,\forall u_{1},u_{2},v\in V:\,|B(u_{1},v)-B(u_{2},v)|\leq M||u_{1}-u_{2}||_{V}\,||v||_{V}$
(...za omezenost)
\item Uniformě monotónní (v 2. složce) $\exists\alpha>0:\,\forall u_{1},u_{2}\in V:\, B(u_{1},u_{1}-u_{2})-B(u_{2},u_{1}-u_{2})\geq\alpha||u_{1}-u_{2}||_{V}^{2}$
(...za elipticitu)
\end{itemize}
Pak pro každou $F\in V^{\star}$ existuje $u\in V$, že $B(u,v)=<F,v>\,(\forall v\in V)$.

{\footnotesize{V obou požadavcích je $u_{1},u_{2}$ v první složce,
protože v druhé složce je záležitost lineární.}}{\footnotesize \par}

\pagebreak{}


\subsection*{Caratheodory mapping}


\paragraph*{Caratheodoryova $f\textrm{ : }\Omega\times\mathbb{R}^{L}\rightarrow\mathbb{R}\Leftrightarrow$}
\begin{itemize}
\item $f(.,u)$ je měřitelná v 1. složce $x$ pro $\forall u\in\mathbb{R}^{L}$
(v případě tříargumentové funkce je $u$ poslední dva argumenty)
\item $f(x,.)$ je spojitá v 2. složce $u$ pro s.v $x\in\Omega$
\end{itemize}

\subsection*{Fundamental theorem of the calculus of variations}


\paragraph*{Zadání:}

$\Omega\in C^{0,1}\textrm{(lipschitzovská), }u_{0}\in W^{1,p}(\Omega,\mathbb{R}^{L})\textrm{, }f\in(W_{0}^{1,p}(\Omega,\mathbb{R}^{L}))^{\star}$

$F\textrm{: }\Omega\times\mathbb{R}^{L}\times\mathbb{R}^{d\times L}\rightarrow\mathbb{R}$
Caratheodoryova


\paragraph*{Variační problém:}

Najít $u,\, u-u_{0}\in W_{0}^{1,p}(\Omega,\mathbb{R}^{L})$ které
minimalizuje funkcionál (na zmíněné zadané množině $v-u_{0}\in W_{0}^{1,p}(\Omega,\mathbb{R}^{L})$).

\[
J(u):=\int_{\Omega}F(x,v,\nabla v)-<f,v-u_{0}>
\]


Respektive - najít $u,\, u-u_{0}\in W_{0}^{1,p}(\Omega,\mathbb{R}^{L})$
, že

\[
\int_{\Omega}F(x,u,\nabla u)\leq\int_{\Omega}F(x,v,\nabla v)-<f,v-u>\,\forall v\in W^{1,p}(\Omega,\mathbb{R}^{L}),\, v|_{\delta\Omega}=u_{0}
\]



\paragraph*{Minimální požadavky:}
\begin{enumerate}
\item P-koercivita $F(x,u,\eta)\geq\alpha|\eta|^{p}+g(x)$, $g\in L^{1}(\Omega)$
\item P-růst $|F(x,u,\eta)|\leq c|\eta|^{p}+g(x)$
\end{enumerate}
{\footnotesize{Pozor, koercivita nemá abs. hodnotu! Pamatovat si třeba
podle třídy ekvivalence s polynomem p. Napsat existenci konstant!
TODOD je to to samy p co v definici zadani?}}{\footnotesize \par}


\paragraph*{Slabá zdola polospojitost}

Pro operátor splňující minimální požadavky definujeme slabou zdola
polospojitost: 
\[
\Leftrightarrow\forall u^{n}\rightharpoonup u\textrm{ v }W^{1,p}(\Omega,\mathbb{R}^{L})\textrm{ platí }\liminf_{n\rightarrow\infty}\int_{\Omega}F(.,u^{n},\nabla u^{n})\geq\int_{\Omega}F(.,u,\nabla u)
\]


{\footnotesize{Proč liminf větší? Protože polospojitost, kdyby celá
spojitost, je omezené i limsup.}}{\footnotesize \par}


\paragraph*{Základní věta variačního kalkulu}

Nechť $F$ ze zadání splňuje minimální požadavky a je slabě zdola
polospojitá.

Pak existuje $u$ řešící variační problém s daným zadáním.


\paragraph*{D}

Vezmeme variační problém a přičteme nulu:

\[
\int_{\Omega}F(x,u,\nabla u)-<f,u-u_{0}>\leq\int_{\Omega}F(x,v,\nabla v)-<f,v-u_{0}>
\]


definujeme 
\[
J(v):=\int_{\Omega}F(x,v,\nabla v)-<f,v-u_{0}>
\]


tím je vidět, že obě formulace jsou totožné, mějme tedy $I=\inf_{v\in W^{1,p}(\Omega,\mathbb{R}^{L}),v|_{\delta\Omega}=u_{0}}J(v)$.
Budeme chtít ukázat, že infimum se nabývá a ověřit variační nerovnost.

Z koercivity plyne, že infimum není $-\infty$: 

\[
J(v)\geq\alpha||\nabla v||_{p}^{p}-c-||f||\,||\nabla v-\nabla u_{0}||_{p}\geq\alpha||\nabla v||_{p}^{p-1}(||\nabla v||_{p}-c)
\]


Z existence infima $I$ máme posloupnost $u^{n},\, I=\lim_{n\rightarrow\infty}J(u^{n})$

Protože jsem v infimu, porovnám s prvkem prostoru, např $u_{0}$.
Pak zmizí část s $f$ a díky p-růstu dostaneme následující nerovnost
(druhou konstantu dostaneme jako normu $g$ a zvětšíme, abychom mohli
konstantu vytknout):

\[
I\leq J(u_{0})=\int_{\Omega}F(x,u_{0},\nabla u_{0})-0\leq c(1+||\nabla u_{0}||_{p}^{p})
\]


...a na druhou stranu ta posloupnost k tomu infimu běží, tedy z koercivity:

\[
I=\lim_{n\rightarrow\infty}J(u^{n})\geq\alpha\liminf_{n\rightarrow\infty}(||\nabla u^{n}||_{p}^{p}-c)
\]


Tedy se jedná o omezenou posloupnost:

\[
||\nabla u^{n}||_{p}\leq c(f,u_{0})
\]


Protože stopa nulová, tak je omezená celá norma $||u^{n}||_{1,p}\leq c(f,u_{0},\Omega)$.
(Je pak na sob. prostoru se zafixovanou stopou ekvivalentní seminorma
a norma. Tedy máme li omezenou seminormu -což máme-, máme omezenou
i sobolevovskou normu.)

Díky tomu z reflexivity dostaneme (búno s přechodem k podposloupnosti)
$u^{n}\rightharpoonup u$ ve $W^{1,p}(\Omega,\mathbb{R}^{L})$. Tohle
$u$ prohlásíme za řešení a ověříme, že tomu tak skutečně je.

Protože je $I$ infimum, tak pro každé $v$ odhaduje funkcionál $I$,
který z infima dával posloupnost $u^{n}$, tedy máme:

\[
\int_{\Omega}F(x,v,\nabla v)-<f,v-u_{0}>\ge I=\lim_{n\rightarrow\infty}J(u^{n})\geq\liminf_{n\rightarrow\infty}J(u^{n})=
\]


...z definice funkcionálu pak rozepíšeme liminf a ze slabé konvergence
a díky předpokládáné slabé zdola polospojitosti dostaneme:

\[
=\liminf_{n\rightarrow\infty}J(u^{n})=\liminf_{n\rightarrow\infty}\int_{\Omega}F(x,u^{n},\nabla u^{n})-<f,u^{n}-u_{0}>\geq\int_{\Omega}F(x,u,\nabla u)-<f,u-u_{0}>
\]


Ověřili jsme pro nalezené $u$, že pro libovolné $v$ je variační
nerovnost splněna.


\subsection*{weak lower-semicontinuity theorem for convex functions}

Nechť $F$ splňuje minimální předpoklady. Navíc, nechť $F$:
\begin{itemize}
\item konvexní v poslední proměnné a
\item $A_{i}^{\nu}(.,u,\eta):=\frac{\partial F(.,u,\eta)}{\partial\eta_{i}^{\nu}}$
je Caratheodoryho a
\item $|A(.,u,\eta)|\leq c(g(x)+|\eta|^{p-1})$ pro nějaké $g\in L^{p'}(\Omega)$
\end{itemize}
Pak F je slabě zdola polospojitá.


\paragraph*{D}

Máme tedy $u^{n}\rightharpoonup u$ v $W^{1,p}(\Omega,\mathbb{R}^{L})$
a chceme $\liminf_{n\rightarrow\infty}\int_{\Omega}F(.,u^{n},\nabla u^{n})\geq\int_{\Omega}F(.,u,\nabla u)$.

Nejprve odvodíme z konvexity nerovnost:
\[
A_{i}^{\nu}(.,u,a)\cdot(b-a)\leq F(.,u,b)-F(.,u,a)
\]


Jak? Začneme definicí konvexity:

\[
F(u,\lambda b+(1-\lambda)a))-F(u,a)\leq\lambda(F(u,b)+(1-\lambda)F(u,a))
\]


\[
F(u,a+\lambda(b-a))-F(u,a)\leq\lambda(F(u,b)-F(ua))
\]


Podělíme $\lambda$ a pošleme do nuly. Pravá strana bude rovná pravé
straně dokazované nerovnosti a levou ještě trochu upravíme:

\[
1/\lambda\,(F(u,a+\lambda(b-a))-F(u,a))=1/\lambda\,\int_{0}^{1}\frac{d}{dt}F(u,a+t\lambda(b-a))\, dt=
\]


...rozderivováním zmizí $\lambda$:

\[
=\sum\int_{0}^{1}\frac{\partial F}{\partial\eta_{i}^{\nu}}(b_{i}^{\nu}-a_{i}^{\nu})\, dt=\sum\int_{0}^{1}A_{i}^{\nu}(b_{i}^{\nu}-a_{i}^{\nu})\, dt
\]


Tedy nerovnost $1/\lambda\, F(u,a+\lambda(b-a))-F(u,a)\leq(F(u,b)-F(u,a))$
přejde limitně k 

\[
A(u,a)\cdot(b-a)\leq F(.,u,b)-F(.,u,a)
\]


...což jsme chtěli dokázat. (Anebo lze říct, že pro konvexní funkci
je derivace neklesající, takže derivace v bodě a je nejmenší z celého
intervalu + věta o střední hodnotě?)

Slabá zdola polospojitost teď vyjde z těchto úprav:

\[
\liminf_{n\rightarrow\infty}\int_{\Omega}F(.,u^{n},\nabla u^{n})\, dx=\liminf_{n\rightarrow\infty}\int_{\Omega}F(.,u^{n},\nabla u^{n})-F(.,u^{n},\nabla u)\, dx+\liminf_{n\rightarrow\infty}\int_{\Omega}F(.,u^{n},\nabla u)\, dx\geq
\]


...dle odvozené konvexní nerovnosti:

\[
\geq\liminf_{n\rightarrow\infty}\int_{\Omega}A(.,u^{n},\nabla u^{n})\cdot(\nabla u^{n}-\nabla u)\, dx+\liminf_{n\rightarrow\infty}\int_{\Omega}F(.,u^{n},\nabla u)\, dx
\]


Když v pravé straně této nerovnosti přejdeme k limitě, pak první člen
půjde k nule (protože $\nabla u^{n}-\nabla u$ jde z předpokladu slabě
k nule v $L^{p}$) a druhý člen jde k $\int_{\Omega}F(.,u,\nabla u)$.

TODOD můžu takhle beztrestně jít k limitě jen na jedný straně nerovnosti?

\pagebreak{}


\subsection*{Věta Nemytskii operator}

Nechť $f:\,\Omega\times\mathbb{R}^{L}\rightarrow\mathbb{R}$ je Caratheodoryova
a předpokládáme, že pro nějaké $1\leq q_{1},...,q_{L},p<\infty$ platí

\[
|f(x,u)|\leq g(x)+\sum_{i=1}^{L}|u_{i}|^{q_{i}/p}\textrm{ s }g\in L^{p}(\Omega)\textrm{ pro s.v. }x\textrm{ a }\forall u\in\mathbb{R}^{L}
\]


Pak zobrazení $(u_{1}(x)...u_{L}(x))\rightarrow f(x,u(x))$ je spojité
z $L^{q_{1}}(\Omega)\times...\times L^{q_{L}}(\Omega)\rightarrow L^{p}(\Omega)$.

{\small{(Na první řádce je $u$ proměnná, pak je to funkce. Nezapomenout
na předpoklady rovnosti.)}}{\small \par}


\paragraph*{D}

Zobrazení $f(x,u(x))$ bude měřitelné v $\Omega$, jako složení spojité
a měřitelné.


\subparagraph*{Norma}

Dále chceme ukázat, že bude mít konečnou $L^{p}$ normu:

\[
\int_{\Omega}|f(x,u(x))|^{p}\leq\int_{\Omega}|g(x)+\sum_{i=1}^{L}|u_{i}|^{q_{i}/p}\,|^{p}\, dx\leq
\]


... všechny členy po roznásobení lze schovat do konstantního násobku
členů s největšími exponenty:

\[
\leq c(p)(\int_{\Omega}|g(x)|^{p}+\sum_{i=1}^{L}|u_{i}|^{q_{i}}\, dx)\leq c(p)(||g(x)||_{p}^{p}+\sum_{i=1}^{L}||u_{i}||_{q_{i}}^{q_{i}})
\]



\subparagraph*{Spojitost}

Poslední věc, která zbývá je spojitost. Budeme chtít pro $u_{i}^{n}\rightarrow u\textrm{ v }L^{q_{i}}(\Omega)$
aby $0=\limsup_{n\rightarrow\infty}\int_{\Omega}|f(x,u^{n}(x))-f(x,u(x))|^{p}$.
Cíl tedy bude najít podposloupnost, kde se limsup nabývá.

Použijeme Lemma: Pokud $g^{n}\rightarrow g$ v $L^{s}(\Omega)$, pak
existuje podposloupnost $g^{n_{k}}$ a $M\in L^{s}$, že $|g^{n_{k}}|+|g|\leq M$
s.v. v $\Omega$.

Tohle lemma nám dá takto omezenou podposloupnost. Tu využijeme k nalezení
majoranty, která Lebegovou větou ospravedlní záměnu limsup a integrálu,
čímž bude dokázáno, že limsup se nabývá.

Máme tedy $u_{i}^{n}\rightarrow u\textrm{ v }L^{q_{i}}(\Omega),\,|u_{i}^{n}|+|u_{i}|\leq M_{i}\in L^{q_{i}}$(búno
s přechodem k podposloupnosti).

Odhad majorantou:

\[
|f(x,u^{n}(x))-f(x,u(x))|^{p}\leq c(|f(x,u^{n}(x))|^{p}+|f(x,u(x))|^{p})\leq
\]


...podobně jako výše při odhadu normy:

\[
\leq c(|g(x)|^{p}+\sum_{i=1}^{L}(|u_{i}^{n}|^{q_{i}}+|u_{i}|^{q_{i}})\leq
\]


...což je dle lemmatu (s nerovností na dva sčítance s exponentem):

\[
\leq c(|g(x)|^{p}+\sum_{i=1}^{L}(|M_{i}(x)|^{q_{i}})
\]


...a to platí skoro všude. Tedy celý výsledek je v $L^{1}$ a je tedy
integrovatelnou majorantou, lze zaměnit v naší podposloupnosti limsup
a integrál a tedy se suprémum nabývá a záležitost je spojitá.


\subsection*{Strictly (+nonstrictly) monotone mapping, Minty method}

Řekneme, že $A:\,\Omega\times\mathbb{R}^{L}\times\mathbb{R}^{L\times d}\rightarrow\mathbb{R}^{L\times d}$
je 
\begin{enumerate}
\item monotónní vzhledem k třetí proměnné $\eta$ $\Leftrightarrow$ pro
s.v. $x\in\Omega\,\forall u\in\mathbb{R}^{L},\,\forall\eta_{1},\eta_{2}\in\mathbb{R}^{L\times d}$:
\[
(A(x,u,\eta_{1})-A(x,u,\eta_{2}))\cdot(\eta_{1}-\eta_{2})\geq0
\]

\item striktně monotónní vzhledem k třetí proměnné $\eta$ $\Leftrightarrow$
pro s.v. $x\in\Omega\,\forall u\in\mathbb{R}^{L},\,\forall\eta_{1},\eta_{2}\in\mathbb{R}^{L\times d}$:
\[
|A(x,u,\eta_{1})-A(x,u,\eta_{2})|\,|\eta_{1}-\eta_{2}|>0
\]

\end{enumerate}

\paragraph{Trik s mátou:}

Máme operátor $A$, operátor $\overline{A}$, který splňuje slabou
formulaci a víme z apriorních odhadů, že pro všechny $w$ platí:

\[
0\leq\int_{\Omega}(\overline{A}-A)(.,u,\nabla w)\cdot(\nabla u-\nabla w)\,\forall w\in W^{1,p}(\Omega,\mathbb{R}^{L})
\]


Když to platí pro všechny, tak volíme speciální volbu $w=u\pm\lambda v,\, v\in W^{1,p}(\Omega,\mathbb{R}^{L}),\,\lambda>0$.
Pak:

\[
0\leq\pm\lambda\int_{\Omega}(\overline{A}-A)(.,u,\nabla u+\lambda\nabla v)\cdot\nabla v
\]


... podělím $\lambda$ a jdu s $\lambda$ k nule:

\[
0\leq\pm\int_{\Omega}(\overline{A}-A)(.,u,\nabla u)\cdot\nabla v
\]


Ze spojitosti Nimického operátoru a lebegovy konvergenční věty (a
$\pm$) plyne, že se rovná nule $0=\int_{\Omega}(\overline{A}-A)(.,u,\nabla u)\cdot\nabla v$,
že se operátory rovnají.

\pagebreak{}


\subsection*{Dodatky}


\subsection*{Tabulky}

Gauss-green-ostrogradsky-per-partes:

\[
\int u(x)v'(x)\, dx=u(x)\, v(x)-\int u'(x)\, v(x)\, dx
\]


\[
\int_{\Omega}\frac{\partial u}{\partial x_{i}}v\, d\Omega=\int_{\Gamma}u\, v\, n_{i}\, d\Gamma-\int_{\Omega}u(x)\frac{\partial v}{\partial x_{i}}\, d\Omega
\]


\[
\int_{\Omega}\nabla u\cdot v\, d\Omega=\int_{\Gamma}u(v\cdot n)\, d\Gamma-\int_{\Omega}u\nabla\cdot v\, d\Omega
\]


...pro $u=1$ dostaneme Gausse ($\nabla\cdot v=div\, v$):

\[
\int_{\Omega}v\cdot n\, d\Gamma=\int_{\Omega}\nabla\cdot v\, d\Omega-\int_{\Omega}\nabla\cdot v\, d\Omega
\]


\[
\int_{\Omega}\nabla u\cdot\nabla v\, d\Omega=\int_{\Gamma}u\nabla v\cdot n\, d\Gamma-\int_{\Omega}u\nabla^{2}v\, d\Omega
\]


Green-Gauss:

\[
-\int_{\Omega}\triangle u\cdot v\, dx=\int_{\Omega}\nabla u\nabla v\, d\Omega-\int_{\delta\Omega}v\nabla u\cdot n\, d\delta\Omega
\]



\paragraph*{Triky}

Elipticita s dirichletem OK (Friedrichsova ekviv. norma), Neumann
s členem $+u$ vyjde, bez členu nutno přidat $\int_{\Omega}u=0$.
Ze ekvivalentni normy $|u|=0=>u=0$ tou podmínkou $\int_{\Omega}u=0$,
jinak jen $\nabla u=0$.

Věta o stopách na členy $u(0)v(0)$. 

Friedrichsova nerovnost ekvivalence pri dirichletovi.

$\int_{\Omega}xv_{x}v_{y}\geq-\int_{\Omega}|xv_{x}v_{y}|$

Eliptické odhady $b\geq-|b|$, vše pak s mínusem. Konkrétně z $a+b\geq a-|b|$
a doufám, že se odečte míň.

$||u||_{2}^{2}\leq1/2||u'||_{2}^{2}$ pro $L^{2}(0,1)$ Gagliardo
Nirenberg.

Základní věta kalkulu trik $u=\int u'$, $u(x)=\int_{0}^{x}u'\, dy$,
$\int_{0}^{x}|a|\leq\int_{0}^{\infty/1...}|a|$.

Holder $||fg||_{1}\leq||f||_{p}||g||_{q}$, $1/p+1/q=1$.

$uu'=(u^{2}/2)'$

$ab\leq\frac{a^{2}}{2\epsilon}+\frac{\epsilon b^{2}}{2}$ $ab\leq\frac{a^{p}}{p}+\frac{b^{q}}{q}$
Youngova nerovnost $1/p+1/q=1$

\pagebreak{}


\paragraph*{Poznámka, nelinearity:}

0) napíšu si rovnici $-div\, A(\nabla u)=f,\, u|_{\delta u}=u_{0}$

1) Vidim, že přímo s tim nic nejde, aproximativní řešení. Galerkinova
aproximace. u ma být W1,p to je složité, místo toho část W1,p aby
nakonec sjednocení částí W1,p. Použiju řešení žijící v linobalu w1...wn,
složí celé W1,p protože budou hustý. A argument - to je pevný bod
``f(a){*}a\char`\"{} un je řešení na uzávěru linobalu w1...wn.

2) Mam un, mají limitu? Z omezen0 posloupnosti v reflexivním lze vzít
slabě konvergentní. Musím dokázat, že un omezené v Reflex pr. Dokážu
to ve W1,p. Proto apriorní odhady - un omezená ve W1,p. Tedy un slabě
k u v W1,p.

3) Ptáme se, jestli rovnice splněná slabě -$div\, A(\nabla u)=f,\, u|_{\delta u}=u_{0}$

$\int A(\nabla u)\nabla v=\int f\nabla v\,\forall v\in W^{1,p}$

Ale nevim jak e to pro moje u chová, to A grad u. Co s tim? nevim
jestli je to pslneny.

Ale z apriornich odhadu dokazu vykoukat nejen un omez, ale i A(nabla
u\_n) omezena pro vsechny n, v Lp', to je zase reflexivni prostor,
takze lze predpokladat, ze A grad un jde slabe k A s pruhem. A vim
pouze ze A s pruhem lezi v Lp' a ze slaby konvergence vim, ze slaba
formulace je splnena pro A s pruhem.

Chci tam ale misto A s pruhem primo A! Pointa bodu 3 je dokazat A=A
s pruhem.

A k dukazu mintyho trik. Monotonie a spojitost A. Pomoci mintyho triku
a jakmile to mam, mam vsechno.


\subsection*{Povídání, účel, todo}

Účel není udělat skripta, ze kterých to člověk pochopí, účel byl udělat
text, kterej pomůže (mě a zbytku ročníku) se věci naučit přesně do
puntíku na zkoušku s odvozením/odůvodněním každýho jednotlivýho kroku.

Proto jsou tu i povídání malým fontem na co nezapomínat u věcí, u
kterých se může stát, že je někdo zapomene (důležitý říct i proč je
ten přepoklad důležitý). A proto jsou tu taky popřený sázecí pravidla,
jenom proto, aby věc byla pochopitelnější (např. rovnost na konci
řádku, pokud se bude v úpravě pokračovat a tři tečky, pokud vysvětluju
úpravu, která probíhá). Častý používání návěští v důkazech má usnadnit
zapamatování si kroků důkazu (a aby si každý mohl jednoduše udělat
mindmapu). Název je volený speciálně tak, aby nepřipomínal klasický
skripta a vyhodil čtenáře z denní rutiny při prvním přečtení.

Pokud se ma tenhle textik seriozne pouzivat i jinde, nez v ZS2014,
tak zbyva udelat:

-k cemu jsou minimalni predpoklady, caratheodoryovskost a omezenost
shora u Konvexni=\textgreater{} slabe zdola polospojita?

-Staci tam to odvozeni? Lze k nerovnosti z konvexniho pouzit ze prostek
onvexni ma neklesavou derivac ia vetu o stradni hodnote?

-sbirku uloh s kucharkama, aproximativnima kucharkama a navodama jak
co resit.

Holi.
\end{document}
